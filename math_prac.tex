\documentclass{article}
\usepackage[T2A]{fontenc}
\usepackage[utf8]{inputenc}
\usepackage{amsmath,amssymb, amsthm}
\usepackage[russian]{babel}
%\usepackage[dvips]{graphicx}
\usepackage{geometry,color,graphicx}
% \usepackage{russcorr}

\newtheorem{lm}{Лемма}
\newtheorem{define}{Определение}
\newtheorem{reckon}{Замечание}
\newtheorem{thm}{Теорема}
\newtheorem{isay}{Утверждение}
% \addto\english{\renewcommand\proofna‌​me{ABC}}

\title{Эвристический метод исследования двухцветных раскрасок гиперграфов}
\author{Валерия Немычникова}
\date{14 декабря 2016}
\begin{document}
	\maketitle
	\section{Постановка задачи и определения}
	
		
		В этой работе рассматривается задача о раскрасках гиперграфов в два цвета.\\
	 \textbf{Гиперграфом} называется пара $H = (V, E)$, где множество вершин $V$ -- произвольное конечное множество, а множество рёбер $E$ -- набор подмножеств $V$. Гиперграф называется \textbf{$n$-однородным}, если мощность каждого его ребра равна $n$. \\
	 Можно ввести для гиперграфов обобщение двудольности следующим образом. Говорят, что граф удовлетворяет \textbf{свойству $B$}, если его вершины можно раскрасить в два цвета таким образом, что все его рёбра неодноцветны. Если у графа много рёбер по сравнению с количеством вершин, то он может не удовлетворять свойству $B$. Этим мотивировано изучение величины $m(n) = \min\{|E(H)| : H n$-однородный и не удовлетворяет свойству $B\}$.\\
	Рассмотрим обобщение свойства $B$: будем говорить, что $H$ обладает свойством $B_k$, если существует такая раскраска вершин графа в два цвета, что каждое ребро содержит хотя бы $k$ вершин каждого цвета. По аналогии с $m(n)$ введём $m_k(n) = \min\{|E(H)| : H n$-однородный и не удовлетворяет свойству $B_k\}$.\\
    Задача оценки $m_k(n)$ классическая. Было получено много интересных результатов, например, асимптотические оценки:
        сюда вставить оценки.

    Эти результаты не дают хороших оценок при малых $n$ и $k$.
    В работе (Лебедевой) приведён алгоритмический способ исследования малых значений параметров. Предлагается сведение задачи к задаче о минимальной системе общих представителей и применяется известный (ссылка на книжку Райгора) жадный алгоритм поиска с.о.п. Однако применение простой эвристики позволяет улучшить некоторые оценки. 

    Далее работа устроена следующим образом. Вначале приводится таблица известных ранее оценок $m_k(n)$ для малых $n$ и $k$, и наши результаты. Затем описывается сведение поиска $m_k(n)$ к системам общих представителей. Далее описан алгоритм оптимизации размера минимальной с.о.п.. В конце приводятся листинги кода, а также непосредственные примеры графов, на которых достигаются заявленные оценки.
	

	\section{Формулировка известных и новых результатов}

    \begin{table}[]
    \caption{Известные оценки $m_k(n)$}
    \label{tabular:timesandtenses}
    \begin{center}
    \begin{tabular}{cccccccccc}
    \hline
    k / n & 2 & 3 & 4 & 5 & 6 & 7 & 8 & 9 & 10 \\
    \hline
    1 & 3 & 7 & 23 & 51 & 147 & 421 & 1339 & 2401 & 7803 \\
    \hline
    2 & - & - & 4 & 7 & 22 & 69 & 318 & 1104 & 2401 \\
    \hline
    3 & - & - & - & - & 3 & 8 & 16 & 90 & 147\\
    \hline
    \end{tabular}
    \end{center}
    \end{table}
	
    
    \begin{table}[]
    \caption{Новая таблица оценок $m_k(n)$}
    \label{tabular:timesandtenses}
    \begin{center}
    \begin{tabular}{cccccccccc}
    \hline
    k / n & 2 & 3 & 4 & 5 & 6 & 7 & 8 & 9 & 10 \\
    \hline
    1 & 3 & 7 & 23 & 51 & 147 & 421 & 1339 & 2401 & 7803 \\
    \hline
    2 & - & - & 4 & 7 & 22 & 69 & 318 & 1104 & 2401 \\
    \hline
    3 & - & - & - & - & 3 & 8 & 16 & 90 & 147\\
    \hline
    \end{tabular}
    \end{center}
    \end{table}
		
        В таблице 1 приведены оценки из работы Лебедевой. В таблице 2 некоторые оценки заменены новыми.

        Приведена также формулировка результатов в виде теоремы:

        \begin{thm}
            $m_k(hui) \leq kosmos.$
        \end{thm}

	\section{Сведение задачи верхней оценки $m_k(n)$ к поиску минимальной системы общих представителей}

        Зафиксируем в $n-$однородном гиперграфе $G$ количество вершин $|V|$. Обозначим через $\chi$ множество всех раскрасок множества вершин в два цвета.\\
        Рассмотрим следующую систему подмножеств множества $\chi$. Зафиксируем произвольное ребро $e$ гиперграфа $G$. Назовём $e$ $k-$плохим в раскраске $A$, если в раскраске $A$ не выполнено, что в $e$ найдётся хотя бы $k$ вершин красного цвета и $k$ вершин чёрного цвета. Включим в выбранное подмножество те раскраски, в которых $e$ плохое. Теперь рассмотрим все возможные рёбра $G$ (то есть просто все $n$-элементные подмножества множества его вершин) и описанным выше образом сопоставим каждому ребру множество раскрасок, в которых это ребро $k$-плохое. Получили систему из ${|V| \choose n}$ подмножеств $2^{|V|}$-элементного множества. Обозначим такую систему за $\mathcal{A}$.\\
        Пусть в $\mathcal{A}$ найдётся система общих представителей. Тогда для каждой раскраски найдётся ребро, которое в этой раскраске $k-$плохое. А это значит, что $G$ не допускает такой раскраски, что в каждом ребре есть хотя бы $k$ вершин каждого цвета, а значит, свойство $B_k$ нарушено. Отсюда следует, что если размер с.о.п. равен $t$, то $m_k(n) \leq t$.
			
    \section{Описание алгоритма}
    Известен жадный алгоритм получения системы общих представителей: делаем несколько итераций, выбирая на каждой итерации такое множество, которое увеличит размер покрытия на наибольшее возможное значение. Остановимся тогда, когда все элементы исходного множества покрыты (тогда нашли с.о.п.) либо если множества закончились (тогда либо в с.о.п. входят все множества системы, либо с.о.п. не существует). Понятно, что эта система общих представителей, вообще говоря, не минимальна. Можно привести соответствующий пример. 
    Применим следующую простую идею. На каждом шаге алгоритма с некоторой вероятностью $p$ вместо жадного выбора множества, которое лучше всего увеличит размер покрытия, выберем случайное подмножество системы из ещё не выбранных.

    <код>

    \section{Листинги кода}

    Код в удобном виде можно найти по ссылке:

    <code>

    \section{Примеры}
    Ниже приведены примеры графов, на которых достигаются оценки. Графы записаны в таком формате: вначале приводятся значения $n, |V|, k, m_k(n)$, далее через запятую перечислены номера рёбер при их лексикографическом упорядочении. Например, ребро под номером 1 соответствует ребру из последних $n$ вершин. 
            
    \begin{thebibliography}{9} 
				
		\bibitem{los} Cherkashin D.D., Kulikov A.B., “On two-colorings of hypergraphs”, Dokl. Math., 83:1 (2011), 68–71
				
	\end{thebibliography}
\end{document}
